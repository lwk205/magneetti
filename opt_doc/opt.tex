\documentclass[a4paper]{article}

\usepackage{amssymb}
\usepackage{euscript}
\usepackage{graphicx}
\graphicspath{{pics/}}
\usepackage{color}
\usepackage{epsfig}
\usepackage{fullpage}

\newcommand{\image}[3]{
\begin{figure}[#1]
\begin{center}
\input{pics/#2.tex}
\caption{\small#3}
\label{image:#2}
\end{center}
\end{figure}
}

\begin{document}
\section {Solenoid optimizations}

The field is calculated in radial coordinates,~$r$ and~$z$.
We need to optimize at least three coil systems in the volume $V_0$: $r<r_0, |z|<z_0$:
\begin{itemize}
\item Homogeneous field produced by main solenoid and compensation coils
\item Gradient field
\item Minimum (quadratic) field
\end{itemize}
For this we calculate mean square difference from the best
constant, linear, or quadratic approximation.

$$ R = \sqrt{\frac{1}{V_0}\int_V(H(r,z) - H_0 - A f(z,r))^2\,dV}
$$

Function $f(z,r)$ can be choosen according with the fitting model:
$f(z,r)=0$ for constant field, $f(z,r)=z$ for gradient field, 
$f(z,r)=z^2-r^2/2$ for quadratic field or even $f(z,r) = z^4 - 3z^2r^2 + 3/8r^2$
for forth-order field.

The $-1/2$ ratio of~$r$ and~$z$ quadratic terms comes from Maxwell equations
for a cylindrically symmetric system (see below).


This can be rewritten as
$$ R^2
 = I_2 - 2H_0 I_1 + H_0^2 + A( A I_{f2} - 2I_{fH} + 2H_0 I_{f1})
$$



where we introduced integrals
$$
I_1 = \frac{1}{V_0}\int_V H(z,r)\, dV,\qquad
I_2 = \frac{1}{V_0}\int_V H(z,r)^2\, dV,\qquad
$$
$$
I_{fH} = \frac{1}{V_0}\int_V f(z,r) H(z,r)\, dV,\qquad
I_{f1} = \frac{1}{V_0}\int_V f(z,r)  \, dV,\qquad
I_{f2} = \frac{1}{V_0}\int_V f(z,r)^2\, dV,\qquad
$$

Parameters $H_0$ and $A$ correspond the minimum of~$R$:
$$
\frac{\partial R}{\partial H_0} = 0,\quad
\frac{\partial R}{\partial A} = 0
$$

$$
H_0 = I_1 - A I_{f1},\qquad
$$
$$
A = (I_{fH} - I_{f1} I_1)/(I_{f2} - I_{f1}^2),\qquad
$$

For coil optimizations we use dimensionless current-independent
parameters:
$R/|H_0|$ for constant field, $R/(z_0|A|)$ for gradient field, and
$R/(z_0^2|A|)$ for quadratic fiels.

Optimisation is done in $z0=r0=10$ cylindrica volume.

{\bf Fixed geometry.} We use main coil with length 104~mm, inner radius
21~mm, and 10 layers of wire in the cylindrical shield with length 140~mm
and radius 29~mm. Wire diameter is 0.084mm (in insulation).

{\bf Compensation coils.} Minimum value in the optimisation is not so
important as minimum width ($0.2$~mm error will break all the
improvements). It's better to start compensation coils from edge of the
main solenoid and make them thin (I use 2 layes) and long -- then the
minimum is wider. Shorter coils have better field homogeneity, but it is
harder to make accurate size. Second layer with different length does not
help much. Compensation coils inside the solenoid is slightly better. I
also hope that it is better to do correct sizes in this case. Adjasting
the quadratic coil can improve the field homogeneity even more.

{\bf Gradient coils} are better if they are wider. It is good to start them
from the edge of the solenoid. Starting from the end of compensation coils
is also fine (\%10 worse). Second layer can improve gradient coils $\approx 2$ times,

{\bf Quadratic coil} just a coil in the center with a certain width.

{\bf Optimization results:}\\
\medskip
\begin{tabular}{lllllll}
Coils & Edge &Length       & Layers & Err 20x20$^{(1)}$  & Err 10x10$^{(2)}$        & Field [G/A] vs r,z [cm] \\\hline
Main  & 52   & 104         & 10     & $6.4\cdot10^{-4}$ & $1.6\cdot10^{-4}$ & $664.317$\\
Comp  & 52   & 13.6        & 2      & $1.0\cdot10^{-5}$ & $6.8\cdot10^{-7}$ & $665.836$\\
Grad  & 52   & 34.2 + 44.6 & 2+2    & $3.4\cdot10^{-3}$ & $2.1\cdot10^{-4}$ & $91.5273\ z$\\
Quad  & -    & 35.2        & 2      & $4.3\cdot10^{-3}$ & $2.5\cdot10^{-4}$ & $104.159 - 147.93 (z^2-r^2/2)$\\
4-th  & 13.3 & 5.9         & 2      & $1.2\cdot10^{-1}$ & $2.7\cdot10^{-2}$ & $35.6534 - 2.2352 f_4$$^{(3)}$\\
\end{tabular}
\medskip

\noindent
(1) in the volume of optimization, $z0=r0=10$~mm.\\
(2) in the inner volume, $z0=r0=5$~mm.\\
(3) $f_4 = z^4 - 3 z^2 r^2 + 3/8 r^4$


\image{h}{fits}{Calculated field profiles at $1$~A current}

\section*{Field distribution in a cylindrical simmetry (why $z^2+r^2/2$)}

Maxwell equations:
$$
\mbox{div} B = 0,\qquad
\mbox{rot} B = 0
$$

In symmetrical case ($\partial B/\partial\phi = 0, B_\phi=0$):
$$
\frac{\partial B_r}{\partial z} = \frac{\partial B_z}{\partial r}, \qquad
\frac{1}{r} \frac{\partial r B_r}{\partial r} + \frac{\partial B_z}{\partial z}=0.
$$

Multiply the first equation by $r$, differentiate by $r$, and then multiply by $1/r$.
Differentiane the second equation by $z$. Then one can exclude $B_r$:
$$
\frac{1}{r} \frac{\partial B_z}{\partial r}
 + \frac{\partial^2 B_z}{\partial r^2}
 + \frac{\partial^2 B_z}{\partial z^2}=0.
$$

For quadratic distribution $B_z = az^2 + br^2$ one have $a=-2b$.

For 4-power distribution $B_z \propto z^4 - 3 z^2 r^2 + 3/8 r^4$.


\end{document}

